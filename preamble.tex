\section*{Vorwort}

Liebe Leserin, lieber Leser! \\
			
Fertig kompiliert, Beta-getestet und integriert liegt es nun in deinen Händen, das neue Liederbuch für Informatiker, Softwaretechniker, IT-ler und alle Interessierten. 
Es erwartet dich eine bunte Mischung von zweiundzwanzig selbst-gedichteten Songtexten, die die \glqq Welt der Informatik\grqq \ thematisieren, aber nicht allzu ernst nehmen. Das Digitale nimmt heute so viel Einfluss auf unsere Leben wie noch nie zuvor -- da ist es doch völlig angebracht, auch einmal unsere Smartphones und Computer, und unsere Erlebnisse mit ihnen, ausgiebig zu besingen.

Entstanden ist das ganze aus dem Wunsch heraus, den damals von der Stuttgarter Informatik-- und Softwaretechnik-Fachschaft herausgegebenen und mittlerweile etwas in die Jahre gekommenen Liederbüchern \emph{Chor Dump} und \emph{Chor Dump 2} einen würdigen Nachfolger zu bescheren.
Da wir mit der Zeit gehen möchten, heißt der Untertitel des Buches nun natürlich nicht mehr \emph{Effiziente Algorythmen}, sondern \emph{Wiederverwendbare Liedermodule}.

Danken möchten wir Jun.-Prof. Dirk Pflüger, der eines Abends auf der Ferienakademie im schönen Sarntal den Anstoß für die Idee gegeben hat; den fleißigen Fachschaftlern, die tollerweise das Drucken übernommen haben; sowie natürlich allen, die kreative Ideen und Texte für die Lieder eingebracht haben. \\

Damit wäre alles geklärt -- ganz viel Spaß beim Singen! \\

\hfill Dominik Schreiber \\

\ \\

\subsection*{Hinweise}

Die Lieder sind soweit möglich thematisch sortiert; Freunde der Softwareentwicklung beginnen am besten ganz am Anfang, während die \glqq Endbenutzer\grqq \ des Liederbuchs insbesondere auf den Seiten 14 bis 27 auf ihre Kosten kommen. Hymnen auf einzelne, bekannte Institutionen aus der IT-Welt (etwa Linux oder Python) folgen ab Seite 28. Abgerundet wird das Buch durch zwei \glqq Bonus-Lieder\grqq \ spezifisch für die Informatik an der Uni Stuttgart.

\emph{Refrains} sind hinterlegt.
Zu \emph{Strophen} und \emph{Bridges} sind die Akkordfolgen in der Regel einmal angegeben, und werden dann implizit wiederholt. Die Lieder sollte man im Ohr haben, da keine Noten abgedruckt sind.

Die Tonart einiger Lieder wurde zugunsten der Spielbarkeit angepasst.

\pagebreak