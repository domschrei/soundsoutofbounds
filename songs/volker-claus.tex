\subsection{Bonus II: Volker Claus}
\songinfo{Reinhard Mey 
	
	\hfill -- Annabell, ach Annabell}{Daniela Nicklas}

\glqq Wenn Sie mich jetzt noch stoppen wollen, \\ \ \ \ \ müssen Sie schnell sein.\grqq \\
\textit{Zu einem Geburtstagsfest von Prof. Volker Claus (2009)} \\

\refrain{
\chord{A}Volker Claus, ach Volker Claus, da \chord{Hm}ist er schon wieder zur Türe raus \\
\chord{E} Er ändert alles for the best, lebt \chord{A}einfach schneller \chord{E}als der Rest \\
\chord{A}Volker Claus, ach Volker Claus, was \chord{Hm}denkt er sich jetzt schon wieder aus? \\
Viel\chord{E}leicht bleibt uns nur ein Moment, be\chord{A}vor er wieder weiterrennt}

Ich \chord{Hm}kenne ihn noch als Dekan, Sie \chord{G}glauben kaum, was nur ein Mann \\
in \chord{D}einer jungen Fakultät be\chord{E}wegen kann – es ist \chord{A}nie zu spät \\
\chord{Hm}Allen hat er oft erklärt, wie das \chord{G}ist mit dem Curricularnormwert \\
Der der \chord{C\#}Wirklichkeit nicht \chord{F\#m}wirklich entspricht, nur zu \chord{H}ändern war der \chord{E}wirklich nicht \\

\refrain{
Volker Claus, ach Volker Claus, wie hielt das nur Frau Volkert aus? \\
Vermutlich hatte sie fünf Kalender und Diktiergeräte mit Endlosbändern. \\
Volker Claus, ach Volker Claus, was denkt er sich jetzt schon wieder aus? \\
Vielleicht bleibt uns noch ein Moment bevor er wieder weiterrennt}

Aus Oldenburg komm ich grad her, da schätzt man Volker Claus auch sehr \\
Als Geburtshelfer für Lehrerlehre, das ist uns eine große Ehre! \\
Dann gab es auch noch eine Zeit, wo ein dickes Buch er schreibt und schreibt \\
und fragt man ihn: was schreibst Du denn da? \glqq Na, den Duden Informatika!\grqq \\

\refrain{Volker Claus, ach Volker Claus, da ist er schon wieder zur Türe raus \\
Ein Lehrer ruft: \glqq Nur einen Moment, bevor bei uns die Schule brennt!\grqq \\
Volker Claus, ach Volker Claus. Was zieht er da aus der Tasche raus? \\
Den Schülerduden, halb so dick, doch voller Ein-, Aus-, Durchblick.}

Kennen Sie das TSP? Das Traveling-Santa-Claus-Problem? \\
Das löst er immer rasend schnell, dabei ist das nicht-polynomiell! \\
Doch nicht nur zu Nikolaus vertrauen die Hörer Dr. Claus \\
eine glatte Eins in der Evaluation in Theorie und Praxis! Wer schafft das schon?? \\

\refrain{Volker Claus, ach Volker Claus. Ich fürchte, jetzt geht mir der Atem aus \\
So vieles könnt ich noch besingen wenn wir nicht hinterm Zeitplan hingen \\
Volker Claus, ach Volker Claus, da ist er schon fast wieder zur Türe raus \\
Doch heute hier, das ist sein Fest, da halten wir ihn einfach fest.}

\vspace*{4cm}

\begin{center}
	\joke{
		Warum tragen Informatiker immer nur Schuhe mit Klettverschluss? \\
		\ \\
		\ \ldots \\
		\ \\
		Sie haben Angst vor Endlosschleifen.}
\end{center}

\pagebreak
